%\documentclass[prl,twocolumn,showpacs,preprintnumbers,superscriptaddress,nofootinbib]{revtex4}
\documentclass[]{article}
\usepackage{mathrsfs}
\usepackage{amsfonts}
\usepackage{amsmath}
\usepackage{array}
\usepackage{verbatim}
%\usepackage{esptopdf}
\usepackage{epsfig}
\usepackage{graphicx}
\usepackage{epstopdf}
\usepackage{graphicx}
\usepackage{ulem} 
\usepackage{color} 
\newcommand{\edit}[1]{\textcolor{red}{#1}} 
\newcommand{\ep}{\epsilon}

\begin{document}
\title{Searching for a Z': Optimizing Cuts}

\author{Joshua Isaacson, Christopher Willis}
%\affiliation{Department of Physics and Astronomy, Michigan State University,
%East Lansing, MI 48824, USA}
%\author{Christopher Willis}
%\affiliation{Department of Physics and Astronomy, Michigan State University,
%East Lansing, MI 48824, USA}

%\begin{abstract}
% When searching for a new particle, the optimal cuts need to be found. Here we present a study preformed for the LHC
%to search for a Z'.
%\end{abstract}

\maketitle

%\section{Introduction}

\section{Proposal}
To optimize the cuts for the search for a Z' at the LHC focusing on the dielectron channel, we will consider psuedo-data generated by Pythia+Delphes for the analysis. The number of
events for the training and testing will be 100k. The number of events that will be used to determine the signifance for the model will be determined by the integrated luminosity.
We will consider LHC8 with 20.3 $fb^{-1}$, with the intention of extending to LHC13 with varying amounts of integrated luminosities.

The model for the Z' that will be used for this study will be the Sequential Standard Model Z'. With the intention of extending the optimization of the cuts to 
also include other Z' models, such as E6 and Left-Right Symmetric Model.

With the data that we obtain, we will pass it through both a Boosted Decision Tree and a Neural Network to optimize the cuts to 
seperate the background from the signal. We will also use the current cuts for the Z' search to analysis the current effectiveness
of these cuts. The cuts on the data that we will consider are: $p_T$ of the leptons, invariant mass cuts, rapidity cuts of the leptons, and photon isolation.

The background to our process includes the SM Drell-Yan, $t\bar{t}$, diboson production, and QCD. We will start with only considering the largest
of these backgrounds in our analysis (SM Drell-Yan), and adding the other processes in later, since the effects of these should be small.
These processes will also be generated through Pythia+Delphes, again with the number 
of events generated corresponding to the integrated luminosity.

After optimizing the cuts, we will look at what the discovery potential is for the Z' for at least the following masses: 2TeV, 3TeV, 3.5TeV, and 4TeV. However, this range
may change based upon the results of the optimization. The discovery potential will be determined using a log-likelihood ratio of the signal+background hypothesis vs. the
background only hypothesis. We will report the p-values and the standard deviations away from the background hypothesis as a function of invariant mass and integrated
luminosity. The pdf that we will assume the data follows is Possion, and will use this in our determinations of the log-likelihood.

For the Neural Network, we will begin with a single-layer preceptron and add an additional layer if necessary. For the Boosted Decision Tree, we will use TMVA from ROOT. 

\end{document}
